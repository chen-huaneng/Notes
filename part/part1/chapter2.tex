\chapter{Traveling Salesman Problem with Drone}

\section{Flying Sidekick Traveling Salesman Problem}
Flying Sidekick Traveling Salesman Problem (FSTSP) 由Murray(2015)等\cite{murrayFlyingSidekickTraveling2015}提出。FSTSP描述:假定有$c$个顾客需要服务,这个服务可以由无人机或者卡车来提供,但是有些顾客由于某些限制(比如包裹重量超过无人机的载重限制)只能由卡车提供服务。卡车和无人机必须从单一仓库出发,并且返回该仓库一次,即不能重复访问该仓库,无人机和卡车可以同时或者分别离开(返回)仓库。在无人机起飞前需要卡车司机装载包裹和更换电池的时间$s_L$,降落时需要给无人机装卸货物和充电的时间$s_R$。每次无人机的一次服务称为一次sortie,一次sortie分为三个节点$\langle i,j,k \rangle$,起飞点$i$可以是仓库也可以是顾客点,中间节点$j$是需要服务的顾客节点,节点$k$可以是仓库也可以是卡车所在的顾客节点,无人机在整个运输过程中可以进行多次sortie服务多个顾客,但是一次sortie的时间必须在无人机的续航时间内。FSTSP的目标是最小化服务所有顾客并且返回仓库(无人机和卡车都返回)的时间。

关键假设如下:
\begin{itemize}
    \item 无人机每次sortie的过程中只能服务一个顾客节点,但是在这期间卡车可以服务多个顾客节点。
    \item 无人机被假定为匀速飞行,如果无人机比卡车提前到达汇合点则无人机不能在中途停下休息以节省电量。
    \item 无人机可以在降落点重新发射,但是无人机不能返回上一次的发射点。
    \item 无人机和卡车的汇合点必须在顾客节点,而不能在中间的任何位置汇合,并且卡车不会重新访问已经服务过的顾客节点来回收无人机。
    \item 无人机和卡车都不能访问除了仓库以外的非顾客节点(即只考虑简化过后的实际情况),并且无人机和卡车都不能重新访问已经服务过的顾客节点。
    \item 如果无人机返回仓库则不能再次进行服务,这是基于大多顾客节点都离仓库较远(大于无人机的续航里程)的假设,在无人机可以直接起飞服务顾客节点的假设下,PDSTSP会更加适合。
\end{itemize}

FSTSP数学模型的符号含义如表\ref{tab:fstsp-sign-meaning}所示。

\begin{table}[!htbp]
    \begin{threeparttable}
    \centering
    \caption{FSTSP模型符号及含义}
    \label{tab:fstsp-sign-meaning}
    \begin{tabularx}{\textwidth}{lX}
        \toprule[1pt] % 表头线宽1镑(point, pt)
        符号 & 含义 \\
        \midrule[0.75pt] % 表中间线宽0.75镑(point, pt)
        $0$ & 起点仓库 \\
        $c + 1$ & 终点仓库(和起点仓库相同,只是为了建模方便的另一个记号) \\
        $C=\{1,2,\cdots,c\}$ & 全部客户集合 \\
        $C' \subseteq C$ & 无人机可访问的客户集合 \\
        $N_0 = \{0,1,2,\cdots,c\}$ & 流出节点集合 \\
        $N_+ = \{1,2,\cdots,c + 1\}$ & 流入节点集合 \\
        $N = \{0,1,2,\cdots,c,c + 1\}$ & 全部节点集合 \\
        \makecell[l]{$\langle i,j,k\rangle \in P,i \in N_0, j \in\{ C': j \neq i\},$\\
        $k \in\{ N_+: k \neq i, k \neq j,\tau_{ij}'+\tau_{jk}'\leq e\}$} & 无人机飞行路径集合 \\
        $\tau_{ij}'/\tau_{ij}, i \in N_0, j \in N_+, i \neq j, \tau_{0, c+1}\equiv 0\hyperlink{tab:fstsp-item-1}{\tnote{a}}$ & 弧$\langle i,j\rangle$的飞行/行驶时间成本 \\
        $s_L/s_R$ & 无人机发射/回收耗时 \\
        $e$ & 无人机续航时长,以单位时间来衡量 \\
        $x_{ij} \in \{0,1\}, i \in N_0, j \in N_+, i\neq j$ & 卡车路由决策变量 \\
        $y_{ijk} \in \{0,1\}, i \in N_0, j \in C, k\in \{N_+: \langle i,j,k \rangle \in P\}$ & 无人机路由决策变量 \\
        $t_i'/t_i\geq 0, i\in N_+, t_0' = t_0 = 0$ & 无人机/卡车有效到达时间戳辅助变量 \\
        $p_{ij} \in \{0,1\}\hyperlink{tab:fstsp-item-2}{\tnote{b}},p_{0j} = 1 ,\forall j \in C$ & 卡车访问次序先后辅助变量(为了确保无人机连续的sortie和卡车访问的顺序一致\hyperlink{tab:fstsp-item-3}{\tnote{c}}) \\
        $1 \leq u_i \leq c + 2, i \in N_+$ & 卡车破子圈辅助变量(和标准TSP的MTZ形式破子圈辅助变量类似\hyperlink{tab:fstsp-item-4}{\tnote{d}}) \\
        \bottomrule[1pt] % 表尾线宽1镑(point, pt)
    \end{tabularx}
    \begin{tablenotes}
        \footnotesize % 设置脚注内容字体大小为\footnotesize
        \item[a] \hypertarget{tab:fstsp-item-1}{}出于完备性的考虑,当只有一个顾客节点的时候,这个顾客将由无人机从仓库直接起飞进行服务。
        \item[b] \hypertarget{tab:fstsp-item-2}{}当卡车访问顾客节点$j\in \{C:j\neq i\}$时,顾客节点$i \in C$已经在之前的某个时间点被卡车访问过了,则$p_{ij} = 1$。
        \item[c] \hypertarget{tab:fstsp-item-3}{}当顾客节点$i$或者$j$仅被无人机服务时,$p_{ij}$的取值就不重要。
        \item[d] \hypertarget{tab:fstsp-item-4}{}$u_i$表示顾客点$i$在卡车访问的路径中的次序,比如$u_5 = 1$表示顾客点$i = 5$是卡车访问路径中的第1个节点,但是不同于TSP,在FSTSP中需要通过约束将无人机服务的顾客点$i$排除在外。
    \end{tablenotes}
    \end{threeparttable}
\end{table}

FSTSP数学模型可以表示为MILP \ref{model:model-fstsp}。

{
\newcommand{\mySubstack}[1]{\mathclap{\substack{#1}}}
\newcommand{\myYellowTag}[1]{\label{#1}\tag{\colorbox{shallow-yellow}{\theequation}}\addtocounter{equation}{1}}
\newcommand{\myRedTag}[1]{\label{#1}\tag{\colorbox{shallow-red}{\theequation}}\addtocounter{equation}{1}}
\newcommand{\myGreenTag}[1]{\label{#1}\tag{\colorbox{shallow-green}{\theequation}}\addtocounter{equation}{1}}
\newcommand{\myPurpleTag}[1]{\label{#1}\tag{\colorbox{shallow-purple}{\theequation}}\addtocounter{equation}{1}}

\begin{model}{FSTSP MILP}{model-fstsp}
\begin{align}
    \min \quad & t_{c + 1}  \label{eq:fstsp-obj}\\
    \text{s.t.} \quad & 
        \sum_{\substack{i\in N_{0}\\i\neq j}}x_{ij}+\sum_{\substack{i\in N_{0}\\ i\neq j}}\!\sum_{\substack{k \in N_{+}\\ \langle i,j,k \rangle \in P}}y_{ijk}=1, \quad \forall j \in C\myGreenTag{eq:fstsp-customer-visit}\\
    \quad & 
        \sum_{j\in N_{+}}x_{0j}=1\myPurpleTag{eq:fstsp-truck-start}\\
    \quad & 
        \sum_{i\in N_{0}}x_{i,c + 1}=1\myPurpleTag{eq:fstsp-truck-return}\\
    \quad & 
        u_i - u_j + 1 \leq (c + 2)(1 - x_{ij}), \quad \forall i \in C, j \in \{N_{+}:j \neq i\}\myPurpleTag{eq:fstsp-truck-mtz}\\
    \quad & 
        \sum_{\substack{i\in N_{0}\\ i \neq j}}x_{ij}=\sum_{\substack{k\in N_{+}\\k\neq j}}x_{jk}, \quad \forall j \in C\myPurpleTag{eq:fstsp-truck-continuity}\\
    \quad & 
        \sum_{\substack{j\in C\\j \neq i}}\!\sum_{\substack{k\in N_{+}\\\langle i,j,k \rangle\in P}}y_{ijk}\leq1, \quad \forall i\in N_{0}\myRedTag{eq:fstsp-drone-land}\\
    \quad & 
        \sum_{\substack{i\in N_{0}\\i\neq k}}\!\sum_{\substack{j\in C\\\langle i,j,k \rangle\in P}}y_{ijk}\leq1, \quad \forall k \in N_{+}\myRedTag{eq:fstsp-drone-takeoff}\\
    \quad & 
        2 y_{ijk} \leq \sum_{\substack{h \in N_0 \\ h \neq i}}x_{hi} + \sum_{\substack{l \in C\\ l \neq k}}x_{lk}, \quad \forall i \in C, j \in \{C: j \neq i\}, k \in \{N_{+}: \langle i,j,k \rangle \in P\}\myYellowTag{eq:fstsp-drone-truck-meet}\\
    \quad & 
        y_{0jk} \leq \sum_{\substack{h \in N_0 \\ h \neq k}}x_{hk}, \quad \forall j \in C, k \in \{N_{+}: \langle 0,j,k \rangle \in P\}\myYellowTag{eq:fstsp-drone-truck-meet-begin}\\
    \quad & 
        u_k - u_i \geq 1 - (c+2)\left(1 - \sum_{\mySubstack{j \in C \\ \langle i,j,k \rangle \in P}}y_{ijk}\right), \quad \forall i \in C, k \in \{N_{+}: k \neq i\}\myYellowTag{eq:fstsp-truck-visit-order}\\
    \quad & 
        t_i' \geq t_i - M\left(1 - \sum_{\substack{j \in C\\ j \neq i}}\sum_{\substack{k \in N_{+}\\ \langle i,j,k \rangle \in P}}y_{ijk}\right), \quad \forall i \in C\myYellowTag{eq:fstsp-truck-drone-takeoff-time-1}\\
    \quad & 
        t_i' \leq t_i + M\left(1 - \sum_{\substack{j \in C\\ j \neq i}}\sum_{\substack{k \in N_{+}\\ \langle i,j,k \rangle \in P}}y_{ijk}\right), \quad \forall i \in C\myYellowTag{eq:fstsp-truck-drone-takeoff-time-2}\\
    \quad & 
        t_k' \geq t_k - M\left(1 - \sum_{\substack{i \in N_0\\ i \neq k}}\sum_{\substack{j \in C\\ \langle i,j,k \rangle \in P}}y_{ijk}\right), \quad \forall k \in N_{+}\myYellowTag{eq:fstsp-truck-drone-land-time-1}\\
    \quad & 
        t_k' \leq t_k + M\left(1 - \sum_{\substack{i \in N_0\\ i \neq k}}\sum_{\substack{j \in C\\ \langle i,j,k \rangle \in P}}y_{ijk}\right), \quad \forall k \in N_{+}\myYellowTag{eq:fstsp-truck-drone-land-time-2}\\
    \quad & 
    \begin{aligned}
        &
            t_k \geq t_h + \tau_{hk} + s_{L}\left(\sum_{\substack{l \in C\\ l \neq k}}\sum_{\substack{m \in N_{+}\\ \langle k,l,m \rangle \in P}}y_{klm}\right) + s_{R}\left(\sum_{\substack{i \in N_0\\ i \neq k}}\sum_{\substack{j \in C\\ \langle i,j,k \rangle \in P}}y_{ijk}\right) - M(1 - x_{hk}), \\
        &\quad
            \forall h \in N_0, k \in \{N_{+}: k \neq h\}\\
    \end{aligned}\myYellowTag{eq:fstsp-truck-time}\\
    \quad & 
        t_{j}'\geq t_{i}'+\tau_{ij}'-M\left(1 - \sum_{\mySubstack{k \in N_{+}\\ \langle i,j,k \rangle \in P}}y_{ijk}\right), \quad \forall j\in C', i \in \{N_0: i \neq j\}\myRedTag{eq:fstsp-drone-time-fly}\\
    \quad &
        t_{k}'\geq t_{j}'+\tau_{jk}'+s_{R}-M\left(1-\sum_{\mySubstack{i\in N_{0}\\ \langle i,j,k\rangle \in P}}y_{ijk}\right), \quad \forall j\in C', k\in \{N_{+}: k \neq j\}\myRedTag{eq:fstsp-drone-time-fly-back}\\
    \quad & 
        t_k' - (t_j' - \tau_{ij}') \leq e + M(1 - y_{ijk}), \quad \forall k \in N_{+}, j \in \{C: j\neq k\}, i \in \{N_0: \langle i,j,k \rangle \in P\}\myRedTag{eq:fstsp-drone-endurance}\\
    \quad & u_i - u_j \geq 1 - (c+2)p_{ij}, \quad \forall i \in C, j \in \{C: j \neq i\}\label{eq:fstsp-drone-truck-aide-1}\\
    \quad & u_i - u_j \leq -1 + (c+2)(1 - p_{ij}), \quad \forall i \in C, j \in \{C: j \neq i\}\label{eq:fstsp-drone-truck-aide-2}\\
    \quad & p_{ij} + p_{ji} = 1, \quad \forall i \in C, j \in \{C: j \neq i\}\label{eq:fstsp-drone-order}\\
    \quad &
    \begin{aligned}
        &
            t_l' \geq t_k' - M\left(3 - \sum_{\mySubstack{j \in C\\ \langle i,j,k \rangle \in P\\ j \neq l}}y_{ijk} - \sum_{\substack{m \in C \\ m \neq i\\ m \neq k \\ m \neq l}}\sum_{\substack{n \in N_{+}\\ \langle l,m,n\rangle \in P\\ n \neq i\\ n\neq k}}y_{lmn}-p_{il}\right)\\
        &\quad
            \forall i \in N_0, k \in \{N_{+}: k \neq i\}, l \in \{C: l\neq i, l\neq k\}
    \end{aligned}\myRedTag{eq:fstsp-drone-sortie-order}\\
    \quad & t_0 = 0\label{eq:fstsp-truck-start-time}\\
    \quad & t_0'= 0\label{eq:fstsp-drone-start-time}\\
    \quad & p_{0j} = 1, \quad \forall j \in C\label{eq:fstsp-drone-start-order}\\
    \quad & x_{ij} \in \{0, 1\}, \quad \forall i \in N_0, j \in \{N_{+}:j \neq i\}\label{eq:fstsp-x-bound}\\
    \quad & y_{ijk} \in \{0, 1\}, \quad \forall i \in N_0, j \in \{C:j \neq i\}, k \in \{N_{+}:\langle i,j,k \rangle \in P\}\label{eq:fstsp-y-bound}\\
    \quad & 1 \leq u_i \leq c+2, \quad \forall i \in N_{+}\label{eq:fstsp-u-bound}\\
    \quad & t_i \geq 0, \quad \forall i \in N\label{eq:fstsp-truck-time-bound}\\
    \quad & t_i' \geq 0, \quad \forall i \in N\label{eq:fstsp-drone-time-bound}\\
    \quad & p_{ij}\in \{0, 1\}, \quad \forall i \in N_0, j \in \{C: j \neq i\}\label{eq:fstsp-drone-order-bound}
\end{align}
\end{model}
}

目标函数\ref{eq:fstsp-obj}追求最小化卡车到达终点仓库$c+1$的有效时间$t_{c+1}$,通过约束\ref{eq:fstsp-truck-drone-land-time-1}和\ref{eq:fstsp-truck-drone-land-time-2}来对齐无人机和卡车最晚到达终点仓库的时间,所以目标函数相当于$\min\{\max\{t_{c+1},t_{c+1}'\}\}$。

约束条件可以分为四类\cite{zhihu-murray2015}:

\begin{itemize}
    \item \colorbox{shallow-green}{客户有关的约束:}约束\ref{eq:fstsp-customer-visit}要求对于任何一位顾客$j$,必须且只能被卡车(或无人机)服务一次。
    \item \colorbox{shallow-purple}{卡车有关的约束:}
    \begin{itemize}
        \item 卡车流平衡约束:约束\ref{eq:fstsp-truck-start}要求卡车必须从起点仓库流出一次,约束\ref{eq:fstsp-truck-return}要求卡车必须从终点仓库流入一次,约束\ref{eq:fstsp-truck-continuity}要求卡车在中间节点满足流入和流出相等的流平衡约束。
        \item 卡车破子圈约束:约束\ref{eq:fstsp-truck-mtz}是MTZ形式的破子圈约束\cite{YunChouorWeiWoYouHuaQianTanLuXingShangWenTiTSPDeQiZhongZhengShuGuiHuaMoXing2022, YunChouorWeiWoYouHua|TSPZhongLiangZhongBuTongXiaoChuZiHuanLuDeFangFaJicallbackShiXianPythonDiaoYongGurobiQiuJie2020},去除了子圈存在的可能,这里$M$取到了$u_i - u_j + 1$的上界$c+2$,$u_i$可以理解为点$i$的访问次序,比如$u_1 = 5$可以理解为点1是从出发点开始,第五个被访问到的点。
    \end{itemize}
    \item \colorbox{shallow-red}{无人机有关的约束:}
    \begin{itemize}
        \item 无人机发射、回收节点流约束:约束\ref{eq:fstsp-drone-land}表示无人机可以从非终点仓库流出,约束\ref{eq:fstsp-drone-takeoff}表示无人机可以从非起点仓库流入。
        \item 无人机访问、回收节点时间戳约束:约束\ref{eq:fstsp-drone-time-fly}表示无人机访问顾客的时间戳应该符合时间逻辑,即不早于起飞时间戳$t_i'$+前往服务顾客点的飞行时长$\tau_{ij}'$\footnote{起飞时间$s_L$没有被包含进来的原因是,当无人机从顾客点$i$起飞时,约束\ref{eq:fstsp-truck-drone-takeoff-time-1}和\ref{eq:fstsp-truck-drone-takeoff-time-2}会使得$t_i' = t_i$,然后约束\ref{eq:fstsp-truck-drone-land-time-1}和\ref{eq:fstsp-truck-drone-land-time-2}会将起飞的时间$s_L$包含在飞行到顾客点$j$的时间内。},约束\ref{eq:fstsp-drone-time-fly-back}表示无人机回到卡车的时间戳应该符合时间逻辑,即不早于访问顾客点的$t_j'$+返回卡车的飞行时长$\tau_{jk}'$+回收无人机用时$s_R$\footnote{这里降落的时间$s_R$必须被包含进来的原因是,卡车可能比无人机更快到达汇合点$k$(原文中举出了一个例子以助于理解\cite{murrayFlyingSidekickTraveling2015})。}。
        \item 无人机电量续航约束:约束\ref{eq:fstsp-drone-endurance}表示无人机的飞行时间不能超过其续航时间,即到达汇合点$t_k'$的有效时间$-$无人机从节点$i$的起飞时间$(t_j'-\tau_{ij}')$(不直接使用$t_i'$是因为$t_i'$不是起飞的时间戳而是无人机到达$i$点的时间戳)要在无人机的续航时间$e$之内。
        \item 无人机飞行次序约束:约束\ref{eq:fstsp-drone-sortie-order}要求无人机对于任意两条路径$\langle i,j,k \rangle$和$\langle l,m,n \rangle$而言,如果无人机先从节点$i$起飞之后的某个时间才从节点$l$($p_{il}=1$)起飞,则无人机必须先完成上一次飞行才能继续下一次飞行($t_l'\geq t_k'$),并且任意两条路径之间无交叉。
    \end{itemize}
    \item \colorbox{shallow-yellow}{无人机和卡车同步有关的约束:}
    \begin{itemize}
        \item 无人机发射、回收点卡车访问约束:约束\ref{eq:fstsp-drone-truck-meet}要求对于非起点发射的无人机($\forall i \in C$),卡车必须经过无人机的起飞点$i$和降落点$k$,约束\ref{eq:fstsp-drone-truck-meet-begin}要求对于从起点仓库起飞的无人机来说,卡车必须经过无人机的降落点。
        \item 无人机访问顾客时卡车访问次序约束:约束\ref{eq:fstsp-truck-visit-order}要求卡车必须先访问无人机的起飞点再访问无人机的降落点。
        \item 无人机发射点时间戳约束:约束\ref{eq:fstsp-truck-drone-takeoff-time-1}和\ref{eq:fstsp-truck-drone-takeoff-time-2}为无人机发射点的有效时间约束,要求无人机在发射节点的有效时间等于卡车在该点的有效时间,共同实现了卡车和无人机在发射节点时间上的对齐。
        \item 无人机回收点时间戳约束:约束\ref{eq:fstsp-truck-drone-land-time-1}和\ref{eq:fstsp-truck-drone-land-time-2}为无人机回收点的有效时间约束,要求无人机在回收节点的有效时间等于卡车在该点的有效时间,共同实现了卡车和无人机在回收节点时间上的对齐。
        \item 卡车访问顾客节点时间戳约束:约束\ref{eq:fstsp-truck-time}要求卡车访问当前顾客点$k$时必须要先将需要起飞的无人机$s_L$发射或者需要降落的无人机$s_R$回收,并且要大于到达顾客点$h$的有效时间戳$t_h$+路径$\langle h,k \rangle$所花费的时间$\tau_{hk}$\footnote{这里假设卡车从$h \in N_0$行驶到$k \in N_+$。}。
    \end{itemize}
    \item 辅助变量和决策变量:
    \begin{itemize}
        \item 卡车访问次序约束:约束\ref{eq:fstsp-drone-truck-aide-1},\ref{eq:fstsp-drone-truck-aide-2}和\ref{eq:fstsp-drone-order}决定了卡车访问次序$p_{ij}$取值的合理性,$u_i$和$p_{ij}$主要用于约束被卡车访问的节点之间的次序,对于被无人机服务的顾客点$i$或者$j$来说,$u_i$和$p_{ij}$的取值并不重要。
        \item 辅助变量及决策变量的初始值和取值范围:约束\ref{eq:fstsp-truck-start-time}和\ref{eq:fstsp-drone-start-time}给定了卡车和无人机有效时间的初始值,约束\ref{eq:fstsp-drone-start-order}规定了起点仓库的访问次序一定在其他所有顾客节点之前,约束\ref{eq:fstsp-x-bound}和\ref{eq:fstsp-y-bound}给定了决策变量的取值范围,约束\ref{eq:fstsp-u-bound}规定了卡车破子圈辅助变量的取值范围,约束\ref{eq:fstsp-truck-time-bound}和\ref{eq:fstsp-drone-time-bound}规定了卡车和无人机的有效时间必须是非负实数,约束\ref{eq:fstsp-drone-order-bound}给定了卡车访问次序辅助变量的取值范围。
    \end{itemize}
\end{itemize}

在约束\cref{eq:fstsp-truck-drone-takeoff-time-1,eq:fstsp-truck-drone-takeoff-time-2,eq:fstsp-truck-drone-land-time-1,eq:fstsp-truck-drone-land-time-2,eq:fstsp-truck-time,eq:fstsp-drone-time-fly,eq:fstsp-drone-time-fly-back,eq:fstsp-drone-endurance}中,$M\geq \max\{t_{c+1}, t_{c+1}'\}$取一个非常大的数,需要大于等于最后到达终点仓库的卡车(无人机)的有效时间。由于无法事先确定最小可接受的$M$值,因此一种方法是用nearest neighbor heuristic来计算一个访问所有顾客节点并返回仓库的时间上限。算法的大致过程:初始化$M = 0$,然后从仓库开始构建卡车路径($i=0$),找到最近的还未访问过的顾客节点$j$,让$M \gets M + \tau_{ij}$,更新$i = j$然后重复这个过程,即不断添加距离最近的未访问过的顾客节点直到所有的顾客都被访问一遍,最终让$M \gets M+\tau_{i,c+1}$,即让卡车返回仓库。

\subsection{Flying Sidekick Traveling Salesman Problem with Multiple Drops}

\section{Parallel Drone Scheduling Traveling Salesman Problem}

Parallel Drone Scheduling Traveling Salesman Problem (PDSTSP) 同样由Murray(2015)等\cite{murrayFlyingSidekickTraveling2015}提出。PDSTSP适用于大量的顾客节点在无人机直接从仓库起飞的续航里程范围内。PDSTSP描述:一辆卡车和一群(单个或者多个都可以)完全相同的无人机从单个仓库出发分别服务顾客,每个顾客只能被服务一次,卡车遵循TSP路径服务,无人机直接从仓库起飞服务顾客,不同于FSTSP,PDSTSP中的无人机不需要和卡车进行汇合。PDSTSP的目标是最小化最终到达仓库的卡车(无人机)的时间。

PDSTSP数学模型的符号含义如表\ref{tab:pdstsp-sign-meaning},基本上沿用了FSTSP的符号。

\begin{table}[!htbp]
    \begin{threeparttable}
    \centering
    \caption{PDSTSP模型符号及含义}
    \label{tab:pdstsp-sign-meaning}
    \begin{tabularx}{\textwidth}{lX}
        \toprule[1pt] % 表头线宽1镑(point, pt)
        符号 & 含义 \\
        \midrule[0.75pt] % 表中间线宽0.75镑(point, pt)
        $0$ & 起点仓库 \\
        $c + 1$ & 终点仓库 \\
        $C=\{1,2,\cdots,c\}$ & 全部客户集合 \\
        $C' \subseteq C$ & 可以接受无人机访问的客户集合\hyperlink{tab:pdstsp-item-1}{\tnote{a}}\\
        $C'' \subseteq C'$ & 在无人机的航行范围内可以接受无人机服务的顾客集合\hyperlink{tab:pdstsp-item-2}{\tnote{b}}\\ 
        $N_0 = \{0,1,2,\cdots,c\}$ & 流出节点集合 \\
        $N_+ = \{1,2,\cdots,c + 1\}$ & 流入节点集合 \\
        $N = \{0,1,2,\cdots,c,c + 1\}$ & 全部节点集合 \\
        $v \in V$ & 无人机集合\\ 
        $\tau_{i,j}'/\tau_{i,j}$ & 弧$\langle i,j\rangle$的飞行/行驶时间成本 \\
        $1 \leq \hat{u}_i \leq c + 2$ & 卡车破子圈辅助变量 \\
        $\hat{y}_{i,v} \in \{0, 1\}, i \in C'', v\in V$ & 无人机访问决策变量\\ 
        $\hat{x}_{i,j} \in \{0, 1\}, i \in N_0, j \in \{N_+:j\neq i\}$ & 卡车路由决策变量\\ 
        \bottomrule[1pt] % 表尾线宽1镑(point, pt)
    \end{tabularx}
    \begin{tablenotes}
        \footnotesize % 设置脚注内容字体大小为\footnotesize
        \item[a] \hypertarget{tab:pdstsp-item-1}{}指包裹重量没有超过无人机的载重限制,不需要顾客签收,顾客的位置允许无人机起降等限制。
        \item[b] \hypertarget{tab:pdstsp-item-2}{}当$\tau_{0,i}'+\tau_{i,c+1}'\leq e$时,顾客$i \in C'$属于集合$C''$。
    \end{tablenotes}
    \end{threeparttable}
\end{table}
\hyperlink{tab:fstsp-item-4}{\tnote{d}}


PDSTSP数学模型可以表示为MILP \ref{model:model-pdstsp}。

{
\newcommand{\mySubstack}[1]{\mathclap{\substack{#1}}}

\begin{model}{PDSTSP MILP}{model-pdstsp}
\begin{align}
    \min \quad & z \label{eq:pdstsp-obj}\\
    \text{s.t.} \quad & z \geq \sum_{\substack{i \in N_0}} \sum_{\substack{j \in N_{+}\\ j\neq i}}\tau_{i,j}\hat{x}_{i,j} \label{eq:pdstsp-truck-min-time}\\
    \quad & z \geq \sum_{i \in C''}(\tau_{0,i}' + \tau_{i, c+1}')\hat{y}_{i,v},\quad \forall v \in V\label{eq:pdstsp-drone-min-time}\\
    \quad & \sum_{\substack{i \in N_0\\ i\neq j}}\hat{x}_{i,j} + \sum_{\substack{v \in V\\ j \in C''}}\hat{y}_{j,v} = 1, \quad \forall j \in C\label{eq:pdstsp-customer}\\
    \quad & \sum_{j \in N_{+}}\hat{x}_{0,j} = 1\label{eq:pdstsp-truck-start}\\
    \quad & \sum_{i \in N_0}\hat{x}_{i,c+1} = 1\label{eq:pdstsp-truck-end}\\
    \quad & \sum_{\substack{i \in N_0\\i \neq j}}\hat{x}_{i,j} = \sum_{\substack{k \in N_{+}\\ k\neq j}}\hat{x}_{j,k}, \quad \forall j \in C\label{eq:pdstsp-truck-flow}\\
    \quad & \hat{u}_i - \hat{u}_j + 1 \leq (c+2)(1 - \hat{x}_{i,j}), \quad \forall i \in C, j \in \{N_{+}:j \neq i\}\label{eq:pdstsp-mtz}\\
    \quad & 1 \leq \hat{u}_i \leq c+2, \quad \forall i \in N_{+}\label{eq:pdstsp-u-bound}\\
    \quad & \hat{x}_{i,j} \in \{0,1\},\quad \forall i \in N_0, j \in \{N_{+}: j\neq i\}\label{eq:pdstsp-x-bound}\\
    \quad & \hat{y}_{i,v} \in \{0,1\},\quad \forall i \in C'', v \in V\label{eq:pdstsp-y-bound}
\end{align}
\end{model}
}

目标函数\ref{eq:pdstsp-obj}追求最小化完工时间$z$,即无人机和卡车最晚到达终点仓库的时间,通过约束\ref{eq:pdstsp-truck-min-time}和\ref{eq:pdstsp-drone-min-time}分别限制卡车和无人机最晚到达终点仓库的时间来实现;约束\ref{eq:pdstsp-customer}确保了每个顾客能且只能被服务一次,服务可以由无人机或者卡车提供;约束\ref{eq:pdstsp-truck-start}和\ref{eq:pdstsp-truck-end}要求卡车必须从起点仓库$0$出发并返回终点仓库$c+1$一次,约束\ref{eq:pdstsp-truck-flow}要求卡车在中间的顾客节点满足流入和流出相等的流约束;约束\ref{eq:pdstsp-mtz}是MTZ形式的破子圈约束;约束\ref{eq:pdstsp-u-bound},\ref{eq:pdstsp-x-bound}和\ref{eq:pdstsp-y-bound}给出了决策变量和辅助决策变量的取值范围。
