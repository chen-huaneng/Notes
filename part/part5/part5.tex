\part{Datasets for Traveling Salesman Problem with Drone}

\chapter{Datasets used in FSTSP-MD}

Traveling Salesman Problem with Drones (TSP-D\index{TSP-D}) 是经典的TSP问题的拓展,它在TSP问题的基础上增加了无人机。无人机可以和车辆一起工作,或者自主起飞服务。根据无人机单次起飞降落过程中服务的顾客点数量的不同可以将问题分为单次起飞服务单个顾客点的和单次起飞服务多个顾客点。同样对无人机和车辆的会合点也有限制,即无人机只能在顾客节点或者仓库节点会合,因此会产生无人机和车辆之间互相等待的时间。接下来的部分将会介绍一些用于TSP-D问题的数据集。

\section{TSP-D Instances by Bouman et al. (2018)}\label{sec:tsp-d-instances-by-bouman-et-al-2018}

\href{https://github.com/pcbouman-eur/TSP-D-Instances?tab=readme-ov-file}{TSP-D-Instances}仓库包含了用于TSP-D问题的二维数据集,即只有仓库和顾客节点的横纵坐标。在数据集中以符号\texttt{/*}开始,以符号\texttt{*/}结束的行是注释行,在读取数据时需要忽略。该数据集包含了Agatz等(2018)\cite{agatzOptimizationApproachesTraveling2018}和Bouman等(2018)\cite{boumanDynamicProgrammingApproaches2018}所用的数据集。相关的解决代码(\texttt{Java}实现)可以在\href{https://github.com/pcbouman-eur/Drones-TSP?tab=readme-ov-file}{Drones-TSP}仓库中找到。在这个数据集中,两点之间的距离是欧几里得距离(Euclidean distance\index{Euclidean distance}),即两点之间的距离是两点之间的直线距离。有关数据集字段说明可以参考数据集的注释和\href{https://github.com/pcbouman-eur/TSP-D-Instances?tab=readme-ov-file}{数据集仓库的说明}。该数据集分类如下(在所有的情况中,生成的第一个节点位置被选作仓库节点):

\begin{itemize}
    \item \texttt{uniform}:每个节点的$x$和$y$坐标都是从取值范围为$\{0,1,\dots,100\}$的独立均匀分布中随机生成的。
    \item \texttt{singlecenter}:对于每个位置,首先均匀地从区间$[0,2\pi]$中抽取一个角度$\alpha$,然后从一个均值为$0$,标准差为$50$的正态分布中抽取一个距离$r$,坐标$(x,y)=(r\cdot \cos \alpha,r\cdot \sin\alpha)$,用这种方法生成的节点位置更有可能集中在中心点$(0,0)$附近,比\texttt{uniform}的数据集更能模拟圆形城市中心的情况。
    \item \texttt{doublecenter}:生成方式和\texttt{singlecenter}类似,但在生成每个位置后,有50\%的概率将其沿$x$轴平移$200$个距离单位,这种方法生成的节点位置更有可能集中在两个中心点$(0,0)$和$(200,0)$附近,模拟了一个具有两个中心的城市的情况。
    \item \texttt{restricted}:在原有限制的基础上增加了一些额外的限制。
    \begin{itemize}
        \item \texttt{maxradius}:增加了无人机不能飞行超过一定半径的限制。
        \item \texttt{novisit}:增加了无人机不能访问的顾客节点,比如以\texttt{-novisit-20-rep\_2.txt}为后缀的文件表示有20\%的顾客节点被随机选中用于表示无法被无人机访问的顾客节点,由于对于同一个数据来说,不同次数的随机生成会影响选中的顾客节点,因此\texttt{rep\_2}表示第二次随机生成的数据。具体的不能被无人机访问的顾客节点由字段\texttt{\#NOVISIT}表示,例如数据文件中的\texttt{\#NOVISIT 1}表示第一个顾客节点不能被无人机访问,也即生成的节点数据中的第二行数据(因为默认生成的第一行数据是仓库节点)。
    \end{itemize}
\end{itemize}

\section{TSPDroneLIB by Bogyrbayeva et al. (2023)}
\href{https://github.com/chkwon/TSPDroneLIB}{TSPDroneLIB}仓库包含了用于TSP-D问题和FSTSP问题的数据集和相关的链接。该仓库提到了\ref{sec:tsp-d-instances-by-bouman-et-al-2018}的数据集,另外包括了Bogyrbayeva等(2023)\cite{bogyrbayevaDeepReinforcementLearning2023}使用的数据集。相关的算法可以在\href{https://github.com/chkwon/TSPDrone.jl}{TSPDrone.jl}仓库中找到。有关数据集的字段说明可以在\href{https://github.com/chkwon/TSPDroneLIB/blob/main/data/Bogyrbayeva/description.md}{TSPDroneLIB/data/Bogyrbayeva/description.md}中找到。该数据集分类如下:

\begin{itemize}
    \item \texttt{Random:}包含了三个不同节点数量大小的数据集,分别为 $n = 20, 50, 100$,每个数据集包含了$100$个算例,每行表示包含横纵坐标的一个算例,遵循格式 $x_1,y_1,x_2,y_2,\dots,x_n,y_n$,即每行的每两个数一组组成一个节点的横纵坐标,同时 $x_1,y_1$表示仓库节点的横纵坐标。该数据集的生成方法为:在$[1100]\times[1100]$的范围内,从均匀分布中随机抽取每个节点的横纵坐标,唯一的例外是仓库节点,其位置分布在$[0,1]\times[0,1]$范围内,这意味着仓库总是位于角落。这种数据集的生成方法和Agatz等(2018)\cite{agatzOptimizationApproachesTraveling2018}使用的数据集生成方法是一致的。
    \item \texttt{Amsterdam:}该数据集的数据格式和\texttt{Random}数据集相同,共有四种不同大小的算例,分别为 $n = 10, 20, 50, 100$。该数据集基于Haider等(2019)\cite{haiderOptimizingRelocationOperations2019}研究中使用的电动汽车(electric vehicle, EV)停车位置数据集。这些位置反映了潜在顾客的位置,因为电动汽车通常停放在城市街道的路边充电器旁。为了适应Bogyrbayeva等(2023)\cite{bogyrbayevaDeepReinforcementLearning2023}的研究,该文章从整个\texttt{Amsterdam}数据集中随机选取仓库和客户节点来创建不同的问题算例。
\end{itemize}

\subsection{DPS algorithm and DRL algorithm implemented in TSPDrone.jl}
\href{https://github.com/chkwon/TSPDrone.jl}{TSPDrone.jl}仓库如前所述,解决了单车辆配备单架无人机的TSP-D问题,实现了Bogyrbayeva等(2023)\cite{bogyrbayevaDeepReinforcementLearning2023}提出的Divide-Partition-and-Search (DPS\index{DPS})算法和Deep Reinforcement Learning (DRL\index{DRL})算法,其中DPS算法是基于Agatz等(2018)\cite{agatzOptimizationApproachesTraveling2018}的TSP-ep-all算法和Poikonen等(2019)\cite{poikonenBranchandBoundApproachTraveling2019}的divide-and-conquer启发式算法开发的。

DPS算法是Bogyrbayeva等(2023)\cite{bogyrbayevaDeepReinforcementLearning2023}提出的一种基于分治策略的启发式算法,用于解决TSP-D问题。其核心思想是将大规模问题分解为较小的子问题,并通过组合子问题的解来获得全局解。DPS算法的主要步骤如下:
\begin{enumerate}
    \item Divide:将全部节点划分为多个子组,每个子组包含固定数量(由参数$g$控制)的节点。例如,$g = 10$表示每个子组有$10$个节点。
    \item Partition \& Search:在每个子组内,采用TSP-ep-all算法\cite{agatzOptimizationApproachesTraveling2018}进行分区。TSP-ep-all算法通过如下步骤优化子组内的路径:
    \begin{enumerate}
        \item 生成初始TSP路径:使用Concorde TSP Solver得到卡车单独服务的初始路径。
        \item 动态规划分区:将路径中的节点划分为由卡车和无人机协同服务的子集,以最小化总时间。
        \item 局部搜索优化:进一步调整分区以提升解的质量。
    \end{enumerate}
    \item 合并与全局优化:将所有子组的解合并为完整的路径,并进行全局优化(如调整子组边界或路径顺序)以消除局部最优的局限性。
\end{enumerate}

在DPS算法中,较大的子组$g$(如$g = 25$)会提升解的质量,但增加计算时间;较小的子组$g$(如$g = 10$)则相反。当$g = N\text{(总节点数)}$时,DPS退化为直接应用TSP-ep-all算法,不再划分子组。

TSPDrone.jl用\texttt{\href{https://julialang.org/}{Julia\index{Julia}}}\footnote{学习\texttt{Julia}的课程参考MIT的\href{https://computationalthinking.mit.edu/}{Introduction to Computational Thinking}。}实现,初始TSP路径由Concorde TSP Solver生成,分区过程基于动态规划和局部搜索,使用该仓库DPS算法的步骤如下:
\begin{enumerate}
    \item 安装\href{https://julialang.org/}{Julia},在命令行中输入\mintinline{pwsh}{julia}进入\texttt{Julia}环境,输入命令安装必要的依赖:
    \mint[bgcolor=Beige,]{julia}{] add https://github.com/chkwon/TSPDrone.jl}
    \item 使用DPS算法需要提供顾客节点的$x$和$y$坐标,仓库的$(x, y)$坐标需要是第一个元素,参数\texttt{truck\_cost\_factor}和参数\texttt{drone\_cost\_factor}分别代表卡车和无人机的成本因子,会乘以从横纵坐标中计算出来的欧氏距离来得到卡车和无人机的行驶成本。

\begin{code}{DPS algorithm usage example (a)}{dps-algorithm-usage-example-a}
    \inputminted{julia}{code/DPS-usage-examples/dps-usage-example-a.jl}
\end{code}
    
    如果正常运行,会输出如下结果(根据随机数生成的结果可能会有所不同),其中节点$11$作为终止节点表示仓库节点(即终止节点的代号会在总的节点数量上$+1$):
\begin{minted}{text}
result.total_cost = 1.6022013835206805
result.truck_route = [1, 4, 5, 2, 8, 6, 11]
result.drone_route = [1, 9, 4, 10, 5, 7, 8, 3, 11]
\end{minted}

    \item 或者也可以直接提供卡车和无人机的成本矩阵(即原本根据欧氏距离矩阵乘以成本因子得到的矩阵),同样,仓库节点被标记为节点$1$:

\begin{code}{DPS algorithm usage example (b)}{dps-algorithm-usage-example-b}
    \inputminted{julia}{code/DPS-usage-examples/dps-usage-example-b.jl}
\end{code}

    \item 使用命令\mintinline{julia}{print_summary(result)}可以输出结果总结:
\begin{minted}{text}
julia> print_summary(result)
Operation #1:
  - Truck        = 0.17988883875173492 : [1, 3]
  - Drone        = 0.11900891950265155 : [1, 4, 3]
  - Length       = 0.17988883875173492
Operation #2:
  - Truck        = 0.4784476248243221 : [3, 9]
  - Drone        = 0.27587675362585756 : [3, 7, 9]
  - Length       = 0.4784476248243221
Operation #3:
  - Truck        = 0.445749847855226 : [9, 6]
  - Drone        = 0.48831605249544785 : [9, 10, 6]
  - Length       = 0.48831605249544785
Operation #4:
  - Truck        = 0.9269158918021541 : [6, 5, 8, 11]
  - Drone        = 0.8714473929102112 : [6, 2, 11]
  - Length       = 0.9269158918021541
Total Cost = 2.073568407873659
\end{minted}
    
    \item 函数\mintinline{julia}{solve_tspd}的可选参数包括:

\begin{code}{DPS algorithm optional keyword arguments}{dps-algorithm-optional-keyword-arguments}
    \inputminted{julia}{code/DPS-usage-examples/dps-optional-keyword-arguments.jl}
\end{code}
    \begin{itemize}
        \item \texttt{n\_groups}:用于分治法的子组数量。例如,如果$n = 100$且\texttt{n\_groups = 4},则每组将有$25$个节点,然后将方法\texttt{method}应用于每个组。
        \item \texttt{method}:可以是以下的几种方法之一,\texttt{TSP-ep}及其衍生方法(\texttt{TSP-ep-1p}、\texttt{TSP-ep-2p}、\texttt{TSP-ep-2opt}和\texttt{TSP-ep-all})是基于route-first, cluster-second框架的启发式算法,由Agatz等(2018)\cite{agatzOptimizationApproachesTraveling2018}提出,用于解决TSP-D:
        \begin{itemize}
            \item \texttt{TSP-ep} (Exact Partition):使用TSP求解器(如Concorde)生成最优TSP路径,然后以初始TSP路径为基础,通过精确划分算法(动态规划,时间复杂度为$O(n^3)$)将TSP路径分割为卡车和无人机的协同路径。
            \item \texttt{TSP-ep-1p}:在\texttt{TSP-ep}的基础上,引入单点移动领域搜索(One-Point Move),通过调整单个节点的位置优化路径。即先对初始路径进行Exact Partition,然后遍历每个节点,尝试将其移动到路径中的其他位置,计算目标函数改进,然后接受最大的移动,迭代直至无法进一步优化。
            \item \texttt{TSP-ep-2p}:在\texttt{TSP-ep}的基础上,引入两点交换领域搜索(Two-Point Swap),通过交换两个节点的位置优化路径。即先对初始路径进行Exact Partition,然后遍历所有节点对,尝试交换两者的位置,计算目标函数改进,然后接受最大的交换,迭代直至无法进一步优化。
            \item \texttt{TSP-ep-2opt}:在\texttt{TSP-ep}的基础上,引入2-opt领域搜索,通过反转路径中的子段优化路径。即先对初始路径进行Exact Partition,然后遍历所有可能的路径子段,尝试反转子段并重新计算总时间,接受改进最大的反转操作,迭代直至无法进一步优化。
            \item \texttt{TSP-ep-all}:在\texttt{TSP-ep}的基础上,综合应用所有领域搜索策略(\texttt{1p}、\texttt{2p}、\texttt{2opt}),通过多策略组合优化路径。即先对初始路径进行Exact Partition,然后在每轮迭代中,尝试所有领域操作(One-Point Move, Two-Point Swap, 2-opt),选择改进最大的操作,迭代直至无法进一步优化。(表现最佳,但运行时间较长$O(n^5)$)
        \end{itemize}
        \item \texttt{flying\_range}:无人机的飞行范围,默认值为\texttt{Inf}。飞行范围与无人机成本矩阵中的值进行比较,即\texttt{drone\_cost\_mtx}或欧氏距离乘以\texttt{drone\_cost\_factor}。
        \item \texttt{time\_limit}:算法运行的总时间限制,以秒为单位。对于每个组,时间限制平均分配。例如,如果\texttt{time\_limit = 3600.0}且\texttt{n\_groups = 5},则每组的时间限制为$3600/5=720$秒。
    \end{itemize}
\end{enumerate}

\section{TSPLIB by Reinelt (1991)}
\href{http://comopt.ifi.uni-heidelberg.de/software/TSPLIB95/}{TSPLIB}是一个用于和TSP问题相关的数据集,包含了Symmetric Traveling Salesman Problem (STSP)\index{STSP}、Asymmetric Traveling Salesman Problem (ATSP)\index{ATSP}、Hamiltonian Cycle Problem (HCP)\index{HCP}、Sequential Ordering Problem (SOP)\index{SOP}、Capacitated Vehicle Routing Problem (CVRP)\index{CVRP}的数据集,可以在\href{http://comopt.ifi.uni-heidelberg.de/software/TSPLIB95/tsp95.pdf}{tsp95.pdf}中查看有关数据集的完整说明文档。除了HCP以外,其他的问题都是定义在完全图上,且所有的距离都是以整数表示的。每个文件都包括说明部分和数据部分,说明部分包含了有关文件的格式和内容的信息。

% 下面介绍和TSPLIB数据集有关的字段说明(符号\texttt{<}和\texttt{>}之间的字符表示数据的类型,比如\texttt{NAME <string>}表示字段\texttt{NAME}的数据类型是\texttt{string}):

% \begin{itemize}
%     \item \texttt{NAME <string>:}数据文件的标识符。
%     \item \texttt{TYPE <string>:}数据文件的类型,包括以下几种:
%     \begin{itemize}
%         \item \texttt{TSP}:Symmetric Traveling Salesman Problem
%         \item \texttt{ATSP}:Asymmetric Traveling Salesman Problem
%         \item \texttt{SOP}:Sequential Ordering Problem
%         \item \texttt{HCP}:Hamiltonian Cycle Problem
%         \item \texttt{CVRP}:Capacitated Vehicle Routing Problem
%         \item \texttt{TOUR}:A collection of tours
%     \end{itemize}
%     \item \texttt{COMMENT <string>:}额外的说明,通常是贡献者或者作者的名字。
%     \item \texttt{DIMENSION <integer>:}对TSP和ATSP来说,表示节点的数量;对于CVRP来说,表示顾客节点和仓库节点的总数;对于\texttt{TOUR}文件来说,表示问题的规模。
%     \item \texttt{CAPACITY <integer>:}对于CVRP来说,表示车辆的容量。
%     \item \texttt{EDGE\_WEIGHT\_TYPE <string>:}表示边的权重(距离)以何种方式给出,包括以下几种类型:
%     \begin{itemize}
%         \item \texttt{EXPLICIT}:边的权重以矩阵的形式给出。
%     \end{itemize}
% \end{itemize}

部分STSP数据集如表\ref{tab:dataset-stsp-example}所示。

\begin{tabularx}{\textwidth}{lccc}
    \caption{Symmetric Traveling Salesman Problem Examples} \label{tab:dataset-stsp-example} \\
    \toprule[1pt]
    数据名称 & 城市数量 & 距离计算方式 & 最优值 \\ 
    \midrule[0.75pt]
    \endfirsthead
    \multicolumn{2}{c}{\tablename\ \thetable (续)} \\ %续表的表头
    \toprule[1pt]
    数据名称 & 城市数量 & 距离计算方式 & 最优值 \\ 
    \midrule[0.75pt]
    \endhead
    \bottomrule[1pt]
    \endfoot
    \bottomrule[1pt]
    \endlastfoot
    % 表格内容
    a280 & 280 & EUC\_2D & 2579 \\
    berlin52 & 52 & EUC\_2D & 7542 \\
    bier127 & 127 & EUC\_2D & 118282 \\
    ch130 & 130 & EUC\_2D & 6110 \\
    ch150 & 150 & EUC\_2D & 6528 \\
    d198 & 198 & EUC\_2D & 15780 \\
    d493 & 493 & EUC\_2D & 35002 \\
    d657 & 657 & EUC\_2D & 48912 \\
    eil51 & 51 & EUC\_2D & 426 \\
    eil76 & 76 & EUC\_2D & 538 \\
    eil101 & 101 & EUC\_2D & 629 \\
    fl417 & 417 & EUC\_2D & 11861 \\
    gil262 & 262 & EUC\_2D & 2378 \\
    kroA100 & 100 & EUC\_2D & 21282 \\
    kroB100 & 100 & EUC\_2D & 22141 \\
    kroC100 & 100 & EUC\_2D & 20749 \\
    kroD100 & 100 & EUC\_2D & 21294 \\
    kroE100 & 100 & EUC\_2D & 22068 \\
    kroA150 & 150 & EUC\_2D & 26524 \\
    kroB150 & 150 & EUC\_2D & 26130 \\
    kroA200 & 200 & EUC\_2D & 29368 \\
    kroB200 & 200 & EUC\_2D & 29437 \\
    lin105 & 105 & EUC\_2D & 14379 \\
    lin318 & 318 & EUC\_2D & 42029 \\
    linhp318 & 318 & EUC\_2D & 41345 \\
    p654 & 654 & EUC\_2D & 34643 \\
    pcb442 & 442 & EUC\_2D & 50778 \\
    pr76 & 76 & EUC\_2D & 108159 \\
    pr107 & 107 & EUC\_2D & 44303 \\
    pr124 & 124 & EUC\_2D & 59030 \\
    pr136 & 136 & EUC\_2D & 96772 \\
    pr144 & 144 & EUC\_2D & 58537 \\
    pr152 & 152 & EUC\_2D & 73682 \\
    pr226 & 226 & EUC\_2D & 80369 \\
    pr264 & 264 & EUC\_2D & 49135 \\
    pr299 & 299 & EUC\_2D & 48191 \\
    pr439 & 439 & EUC\_2D & 107217 \\
    rat99 & 99 & EUC\_2D & 1211 \\
    rat195 & 195 & EUC\_2D & 2323 \\
    rat575 & 575 & EUC\_2D & 6773 \\
    rat783 & 783 & EUC\_2D & 8806 \\
    rd100 & 100 & EUC\_2D & 7910 \\
    rd400 & 400 & EUC\_2D & 15281 \\
    st70 & 70 & EUC\_2D & 675 \\
    ts225 & 225 & EUC\_2D & 126643 \\
    tsp225 & 225 & EUC\_2D & 3919 \\
    u159 & 159 & EUC\_2D & 42080 \\
    u574 & 574 & EUC\_2D & 36905 \\
    u724 & 724 & EUC\_2D & 41910 \\
\end{tabularx}

\section{Amazon Delivery Dataset by Kaggle}
\href{https://www.kaggle.com/datasets/sujalsuthar/amazon-delivery-dataset}{Amazon Delivery Dataset}是一个Amazon公司最后一公里物流运营情况的数据集,包含了超过 $43632$ 次配送的多城市数据,数据字段包括订单详情、配送人员、天气、交通情况、配送仓库和配送地点的经纬度等信息。要将数据集转换为可以用于TSP-D问题的数据集,需要将数据集中的经纬度转换为欧几里得距离,即两点之间的直线距离,当然在此之前需要对原始数据集进行一些数据的预处理工作,详细的代码在附录\ref{sec:amazon-delivery-data-preprocessing-code}中。

