%! TeX program = xelatex
\documentclass[fontset=none]{ctexart}
\input{../note-setup-leftsidebox.tex}

\title{Deep Reinforcement Learning for Multiobjective Optimization}
\author{Chen Huaneng}
\date{\today}

\begin{document}

\maketitle

\section{Deep Reinforcement Learning Based Multiobjective Optimization Algorithm (DRL-MOA)}

这个研究基于两个关键背景:

\begin{itemize}
    \item 多目标优化问题(Multiobjective Optimization Problem, MOP)的基本困境:传统方法(如 NSGA-II, MOEA/D)通过迭代更新种群寻找 Pareto 最优解,但面对大规模问题时(如 200 城市的多目标旅行商问题 MOTSP)时,迭代次数多、计算效率低,且问题稍有变化(如城市位置微调)就需要重新计算;
    \item 深度强化学习(Deep Reinforcement Learning, DRL)的优势:DRL 能通过试错学习训练一个“黑箱模型”,训练后只需要一次前向传播就能输出解,无需迭代,且泛化能力强(能够处理未见过的问题实例)。
\end{itemize}

文章\cite{liDeepReinforcementLearning2021}提出的 DRL-MOA 框架,本质是用“分解思想”(来自 MOEA/D\cite{zhangMOEAMultiobjectiveEvolutionary2007})拆解多目标优化问题,用“DRL + 神经网络”(来自 Pointer Network\cite{vinyalsPointerNetworks2015}、Actor-Critic\cite{nazariReinforcementLearningSolving2018,belloNeuralCombinatorialOptimization2017})求解每个子问题,再用“参数迁移”加速训练。

\section{General Framework}

通用框架是 DRL-MOA 的“骨架”,解决了“如何将多目标问题转化为 DRL 可处理的单目标问题”和“如何高效训练多个子问题的模型”这两个核心问题。分为分解策略和领域参数迁移策略两部分。

\subsection{Decomposition Strategy}

\textit{Decomposition Strategy:} 文章中采用 weighted sum approach \cite{miettinen1999nonlinear} 进行多目标优化问题的分解,也可以采用其他 scalarizing methods,比如 Chebyshev 和 the penalty-based boundary intersection (PBI) method \cite{7572016,7390047}。首先,生成一组均匀分布的权重向量(uniformly spread weight vector)$\lambda^1, \lambda^2, \ldots, \lambda^N$,其中 $N$ 为子问题的数量,比如对于双目标问题( $M = 2$),可以取权重向量为 $(1,0), (0.9,0.1), \ldots, (0,1)$,每个向量表示对不同目标的“重视程度”。对第 $j$ 个权重向量 $\lambda^j = (\lambda^j_1, \lambda^j_2, \ldots, \lambda^j_M)^{\mathrm{T}}$,$M$ 表示目标函数的个数,通过 weighted sum approach,可以将 MOTSP 分解为 $N$ 个单目标优化子问题(scalar optimization subproblems)。第 $j$ 个子问题的目标函数为:

\begin{equation}
    \min g^{ws}(x\mid \lambda^j_i) = \sum_{i=1}^{M} \lambda^j_i f_i(x)\label{eq:obj}
\end{equation}

其中 $f_i(x)$ 是原 MOP 的第 $i$ 个目标函数,$g^{ws}(x\mid \lambda^j_i)$ 是第 $j$ 个子问题的“加权和成本”(单目标)。 

分解后每个子问题的解都是原 MOP 的 Pareto 最优解,这是因为权重向量的不同权衡,使得每个子问题的最优解对应 PF(Pareto Front)上的一个“权衡点”。通过将 MOP 分解成子问题,可以将每个子问题的“加权和成本”作为 DRL 的 “奖励信号(比如奖励 = -加权和成本,因为 DRL 通常最大化奖励,而 MOP 需要最小化成本)。这样就通过将 MOP 拆解为多个标量子问题,每个子问题对应一个“权重向量”,求解所有子问题的解就可以组成 PF。

\subsection{Neighborhood-Based Parameter-Transfer Strategy}

\textit{Neighborhood-Based Parameter-Transfer Strategy:}采用领域参数迁移的策略的核心在于,如果每个子问题都“从头训练”一个神经网络,计算量会非常大($N$ 个子问题需要 $N$ 次独立训练)。但文章根据公式\cref{eq:obj}和Zhang的研究\cite{zhangMOEAMultiobjectiveEvolutionary2007}发现,相邻权重向量对应的子问题,其最优解和最优模型参数非常相似,比如在双目标问题中,权重向量为 $(0.8,0.2)$和$(0.7,0.3)$ 的子问题对于目标的权衡接近,最优路径和模型参数也接近。因此,借鉴 MOEA/D 的“领域更新”思想\cite{zhangMOEAMultiobjectiveEvolutionary2007},提出了领域参数迁移策略,即用前一个子问题的最优模型参数,作为当前子问题的初始参数,避免从头训练,减少计算成本。

其具体过程为:假设已经训练好第 $i - 1$ 个子问题的最优模型参数 $[w_{\lambda^{i-1}}^*, b_{\lambda^{i-1}}^*]$($w$ 为权重,$b$ 为偏置),在训练第 $i$ 个子问题时,使用 $[w_{\lambda^{i-1}}^*, b_{\lambda^{i-1}}^*]$ 作为初始参数($[w_{\lambda^i}, b_{\lambda^i}] = [w_{\lambda^{i-1}}^*, b_{\lambda^{i-1}}^*]$)进行训练。然后在此基础上用 Actor-Critic 进行微调,快速收敛到第 $i$ 个子问题的最优参数 $[w_{\lambda^i}^*, b_{\lambda^i}^*]$。重复该过程,直到所有 $N$ 个子问题都训练完毕。

领域参数迁移策略的优势在于无需为每个子问题初始化随机参数,从而减少了收敛时间,降低了训练复杂度;同时,由于相邻子问题的模型参数平滑过渡,避免 PF 上出现“跳跃”的解,保证了解的一致性和多样性。

\subsection{Pseudo Code of General Framework of DRL-MOA}

DRL-MOA 的通用框架伪代码如\cref{alg:DRL-MOA}所示。每个子问题的训练核心是 Actor-Critic 算法,负责将子问题的“加权和成本”转化为模型的优化信号。训练完成之后,对于新的 MOP 实例,只需要一次前向传播(forward propagation)就能得到对应的 Pareto 最优解,无需重新训练。

\begin{algorithm}[H]
    \KwIn{The model of the subproblem $\mathcal{M} = [\mathbf{w}, \mathbf{b}]$, weight vectors $\lambda^1, \dots, \lambda^N$}
    \KwOut{The optimal model $\mathcal{M}^* = [\mathbf{w}^*, \mathbf{b}^*]$}
    \SetAlgoLined
    \SetNoFillComment
    \vspace{3mm}
    $[\omega_{\lambda^1}, \mathbf{b}_{\lambda^1}] \leftarrow \text{Random\_Initialize}$\;
    \For{$i \leftarrow 1$ \KwTo $N$} {
        \uIf{$i == 1$} {
            $[\omega_{\lambda^1}^*, \mathbf{b}_{\lambda^1}^*] \leftarrow \text{Actor\_Critic}([\omega_{\lambda^1}, \mathbf{b}_{\lambda^1}], g^{\text{ws}}(\lambda^1))$\;
        }
        \Else {
            $[\omega_{\lambda^i}, \mathbf{b}_{\lambda^i}] \leftarrow [\omega_{\lambda^{i-1}}^*, \mathbf{b}_{\lambda^{i-1}}^*]$\;
            $[\omega_{\lambda^i}^*, \mathbf{b}_{\lambda^i}^*] \leftarrow \text{Actor\_Critic}([\omega_{\lambda^i}, \mathbf{b}_{\lambda^i}], g^{\text{ws}}(\lambda^i))$\;
        }
    }
    \Return $[\mathbf{w}^*, \mathbf{b}^*]$\;
    \tcc{Given inputs of the MOP, the PF can be directly calculated by $[\mathbf{w}^*, \mathbf{b}^*]$.}
    \caption{General Framework of DRL-MOA}
    \label{alg:DRL-MOA}
\end{algorithm}

\section{Modeling the Subproblem of MOTSP}

文章的实验实例是多目标旅行商问题 MOTSP:The multiobjective traveling salesman problem (MOTSP), where given $n$ cities and $M$ cost functions to travel from city $i$ to $j$, one needs to find a cyclic tour of the $n$ cities, minimizing the $M$ cost functions. 

\subsection{Formulation of MOTSP}

One needs to find a tour of $n$ cities, that is, a cyclic permutation $\rho$, to minimize $M$ different cost functions simultaneously. 

\begin{equation}
    \min z_k(\rho) = \sum_{i=1}^{n - 1} c^k_{\rho(i), \rho(i + 1)} + c^k_{\rho(n), \rho(1)}, \quad k = 1, 2, \ldots, M
\end{equation}

where $c^k_{\rho(i),\rho(i + 1)}$ is the $k$-th cost of traveling from city $\rho(i)$ to $\rho(i + 1)$. The cost functions may, for example, correspond to tour length, safety index, or tourist attractiveness in practical applications.

\bibliography{references}

\end{document}

