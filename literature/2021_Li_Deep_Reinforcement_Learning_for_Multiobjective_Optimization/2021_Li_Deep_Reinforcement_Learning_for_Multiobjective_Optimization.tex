%! TeX program = xelatex
\documentclass[fontset=none]{ctexart}
%===================================
% note-setup-leftsidebox.tex
% huanengchen@foxmail.com 2025-08-12
%===================================
% 参考:https://tex.stackexchange.com/questions/59702/suggest-a-nice-font-family-for-my-basic-latex-template-text-and-math

%===================================
% 页面和间距
%===================================

\usepackage[a4paper, margin=1in]{geometry} % 具体设置参考 geometry 宏包
\setlength{\parindent}{0pt} % 取消首行缩进
\usepackage{parskip} % 形成段落间的间距
\linespread{1.25} % 修改行距

%===================================
% 编辑体验
%===================================

\usepackage{float} % 优化浮动体
\usepackage[shortlabels,inline]{enumitem} % 优化列表
\usepackage{appendix} % 优化附录

%===================================
% 表格
%===================================

\usepackage{booktabs, multirow, multicol}
\usepackage{tabularx} % 表格自动换行,调整表格宽度 
\usepackage{makecell} % 单元格内换行
\usepackage{threeparttable} % 给表格添加脚注,参考:https://tex.stackexchange.com/questions/6090/clickable-table-footnote
\usepackage{ltablex} % 跨页表格

%===================================
% 参考文献
%===================================

\usepackage[sort&compress]{gbt7714} % 参考文献样式
\bibliographystyle{gbt7714-numerical} % 顺序编码制

%===================================
% 颜色
%===================================

\usepackage[dvipsnames, x11names, table]{xcolor} % 参考:https://tex.stackexchange.com/questions/659036/option-selecting-named-colours-provided-by-the-xcolor-package

%===================================
% 支持插入图片及子图
%===================================

\usepackage{graphicx}
\graphicspath{
    {./figure/}{./figures/}{./image/}{./images/}{./graphic/}{./graphics/}{./picture/}{./pictures/}
} % 用于存放图片的目录,这样引用图片的时候就不需要指定目录
\usepackage{subcaption}

%===================================
% 算法和伪代码
%===================================

\usepackage[linesnumbered, ruled, longend, lined]{algorithm2e} % 参考 algorithm2e 宏包文档
\DontPrintSemicolon % 不打印分号
\setlength{\algomargin}{2em} % 设置算法缩进使得行号在线框内
\renewcommand{\CommentSty}[1]{\normalsize\textit{#1}} % 设置注释的字体样式为意大利斜体,字体大小为 \normalsize

%===================================
% 代码
%===================================

\usepackage{minted} % 参考 minted 宏包文档

% 代码行号样式
\renewcommand{\theFancyVerbLine}{
\sffamily
\textcolor{gray}{
\footnotesize\oldstylenums{
\arabic{FancyVerbLine}}}}

% 行间代码环境
\setminted{
    style=colorful, % 设置代码风格,可选的代码风格参考:https://pygments.org/styles/
    numbers=left, % 显示行号
    numbersep=2pt, % 行号与代码的距离
    mathescape, % 允许在代码注释中使用数学公式
    breaklines, % 允许代码自动断行 
    fontsize=\footnotesize, % 设置代码字体大小
    frame=single, % 设置代码框
    framerule=0.5pt, % 设置代码框线宽
    resetmargins, % 重置代码边距
}

% 行内代码环境
\setmintedinline{
    style=colorful, % 设置代码风格,可选的代码风格参考:https://pygments.org/styles/
    fontsize=\footnotesize, % 设置代码字体大小
    breakanywhereinlinestretch=0.01em, % 允许行内代码在任意位置断行
    breaklines, % 允许行内代码自动断行
}

%===================================
% 超链接
%===================================

\usepackage{hyperref}
\hypersetup{
    bookmarksopen=true, % 启用书签
    colorlinks=true, % 启用颜色
    linkcolor=red, % 内部链接的颜色
    linktoc=all, % 设置目录中的页码和标题都能够跳转
    citecolor=violet, % 引用链接的颜色
    urlcolor=magenta, % 外部链接的颜色
}

% 自定义 autoref 的引用格式
\def\figureautorefname{图} % 将 "Figure" 改为 "图"
\def\tableautorefname{表}  % 将 "Table" 改为 "表"
\def\equationautorefname{公式} % 将 "Equation" 改为 "公式"

%===================================
% 数学公式
%===================================

\usepackage{amsmath, amsthm, amsfonts, amssymb} % 用于加载数学公式、花体字母和数学字符
\usepackage{mathtools}
\usepackage{mathrsfs}
\usepackage{bm}
\usepackage{extarrows}
\usepackage[noabbrev]{cleveref} % 多公式引用,必须放在 hyperref 宏包的后面,参考:https://tex.stackexchange.com/questions/314217/how-i-can-refer-multiple-equation-in-latex

\allowdisplaybreaks[1] % 多行公式换页,1 为尽量避免换页
\crefname{equation}{}{} % 设置非首字母大写的引用格式
\Crefname{equation}{}{} % 设置首字母大写的引用格式
\crefrangeformat{equation}{(#3#1#4)-(#5#2#6)} % 多公式引用的格式

%===================================
% 边注
%===================================

% 设置边注的字体大小
\let\oldmarginpar\marginpar
\renewcommand{\marginpar}[1]{\oldmarginpar{\footnotesize #1}}

%===================================
% 页脚和页眉
%===================================

\usepackage{fancyhdr} % 参考:https://tex.stackexchange.com/questions/732462/chapter-number-in-the-header-with-chapter/732464?noredirect=1#comment1824660_732464 
\usepackage{lastpage} % 获取总页码
\setlength{\headheight}{13pt} % 设置页眉高度

% 重新定义 \author 和 \date 命令用于页眉
\makeatletter
\let\oldauthor\author
\renewcommand{\author}[1]{\oldauthor{#1}\def\myauthor{#1}}
\let\olddate\date
\renewcommand{\date}[1]{\olddate{#1}\def\mydate{#1}}
\makeatother

% 详细参数参考 fancyhdr 宏包
\pagestyle{fancy} % 设置文档的页面样式为 fancy,这意味着页眉和页脚将使用 fancyhdr 宏包提供的自定义格式
\fancyhf{} % 清空原本的页脚页眉样式

% 自定义页眉
\fancyhead[L]{\myauthor} % 左侧显示作者
\fancyhead[R]{\mydate} % 右侧显示日期

\cfoot{\thepage\ / \pageref*{LastPage}} % 自定义页脚,参考:https://tex.stackexchange.com/questions/227/how-can-i-add-page-of-on-my-document

%===================================
% 字体
%===================================

\usepackage[T1]{fontenc} % 改善文档中西欧语言的显示效果
\usepackage{anyfontsize}
\usepackage{lmodern} % lmodern字体
\usepackage{libertine} % Linux Libertine 字体系列(衬线字体)

% 中文字体,具体设置参考 ctex 宏包
\setCJKmainfont{LXGWWenKaiScreen.ttf}[ % 设置中文主字体为霞鹜文楷屏幕舒适版
    Path=../fonts/,
    BoldFont=FZHTJW.TTF, % 设置粗体为方正黑体
    ItalicFont=FZKTJW.TTF, % 设置斜体为方正楷体
]
\setCJKsansfont{FZHTJW.TTF}[ % 设置无衬线字体为方正黑体
    Path=../fonts/,
    AutoFakeBold=1.5,  % 生成粗体效果
    ItalicFont=FZFSJW.TTF, % 设置斜体为方正仿宋
]
\setCJKmonofont{LXGWWenKaiMonoScreen.ttf}[ % 设置等宽字体为霞鹜文楷等宽屏幕舒适版
    Path=../fonts/,
    AutoFakeBold=1.5,  % 生成粗体效果
    ItalicFont=FZSSJW.TTF, % 设置斜体为方正书宋
] 

% 英文字体,具体设置参考 fontspec 宏包
\setmonofont{MapleMono-NF-CN}[ % 设置英文等宽字体
    Path=../fonts/, % 指定字体文件所在的目录
    Extension=.ttf, % 字体文件后缀
    UprightFont=*-Regular, % 正常字体
    BoldFont=*-Bold, % 加粗
    ItalicFont=*-Italic, % 斜体
    BoldItalicFont=*-BoldItalic, % 粗斜体
]

%===================================
% 定理盒子
%===================================

% 修改 proof 环境的引导词为 Proof,样式为加粗无斜体
\renewcommand*{\proofname}{\normalfont\bfseries Proof}

% 导入 thmtools 宏包,使用 \declaretheorem 命令来定义各种定理环境(比 \newtheorem 命令更加方便)
\usepackage{thmtools}

% 定义环境使用的 `\declaretheorem` 命令参数包括:
% - `style`: 定理环境样式,amsthm 内置的样式包括
%   - plain(默认):引导词是正体,内容是斜体
%   - definition:引导词和内容都是正体
%   - remark:引导词是斜体,内容是正体
% - `name`:显示在正文中的引导词(不等于环境的名称)
% - `numbered`:是否开启编号
% - `numberwithin`、`sibling`:定义编号规则,例如:
%   - `numberwithin=section`:基于 section 编号
%   - `sibling=theorem`:共享 `theorem` 环境的编号

% 采用 plain 样式,定义 `theorem`/`theorem*`、`proposition`/`proposition*`、`corollary`/`corollary*`、`lemma`/`lemma*`、`claim`/`claim*` 环境

\declaretheorem[style=plain, name=Theorem, numbered=yes, numberwithin=section]{theorem}
\declaretheorem[style=plain, name=Theorem, numbered=no]{theorem*}

\declaretheorem[style=plain, name=Proposition, numbered=yes, sibling=theorem]{proposition}
\declaretheorem[style=plain, name=Proposition, numbered=no]{proposition*}

\declaretheorem[style=plain, name=Corollary, numbered=yes, sibling=theorem]{corollary}
\declaretheorem[style=plain, name=Corollary, numbered=no]{corollary*}

\declaretheorem[style=plain, name=Lemma, numbered=yes, sibling=theorem]{lemma}
\declaretheorem[style=plain, name=Lemma, numbered=no]{lemma*}

\declaretheorem[style=plain, name=Claim, numbered=yes, sibling=theorem]{claim}
\declaretheorem[style=plain, name=Claim, numbered=no]{claim*}

% 采用 definition 样式,定义 `definition`/`definition*`、`example`/`example*`、`problem`/`problem*` 环境

\declaretheorem[style=definition, name=Definition, numbered=yes, numberwithin=section]{definition}
\declaretheorem[style=definition, name=Definition, numbered=no]{definition*}

\declaretheorem[style=definition, name=Example, numbered=yes, numberwithin=section]{example}
\declaretheorem[style=definition, name=Example, numbered=no]{example*}

\declaretheorem[style=definition, name=Problem, numbered=yes, numberwithin=section]{problem}
\declaretheorem[style=definition, name=Problem, numbered=no]{problem*}

% 采用 remark 样式,定义 `remark`/`remark*`、`note`/`note*` 环境

\declaretheorem[style=remark, name=Remark, numbered=yes, numberwithin=section]{remark}
\declaretheorem[style=remark, name=Remark, numbered=no]{remark*}

\declaretheorem[style=remark, name=Note, numbered=yes, numberwithin=section]{note}
\declaretheorem[style=remark, name=Note, numbered=no]{note*}

% 使用 `\declaretheoremstyle` 命令定义新的 solutionstyle 样式,类似 proof 环境,但是引导词变成 Solution
\declaretheoremstyle[headfont=\bfseries, bodyfont=\normalfont, spaceabove=3pt, spacebelow=3pt, qed=\ensuremath{\square}]{solutionstyle}

% 采用新定义的 solutionstyle 样式,定义 `solution`/`solition*` 环境
\declaretheorem[style=solutionstyle, name=Solution, numbered=yes, numberwithin=section]{solution}
\declaretheorem[style=solutionstyle, name=Solution, numbered=no]{solution*}

% 导入 tcolorbox 宏包以使用盒子美化现有的定理环境
\usepackage[most]{tcolorbox}

% tcolorbox 宏包的功能非常复杂,这里只需要使用 `\tcolorboxenvironment` 命令
% 首先封装一个 `\newtcbenvironment` 命令
% 它可以同时为 `#1` 以及 `#1*` 这两个环境加上盒子,公共参数:
% - `#2`:在定义时传入的参数,这里主要是边框颜色和背景色
% - `enhanced`:样式增强
% - `breakable`:允许跨页
% - `boxrule=1pt`:边框宽度为 1pt
%
% 还有不同的参数:
% - `#1` 盒子使用直角边框(`sharp corners`)
% - `#1*` 盒子使用圆角边框(`rounded corners`)
%
% > 对 `\newtcbenvironment` 内部的公共参数部分进行调整,就可以实现所有盒子只保留左侧边框或者四周无边框等不同的效果。

\newcommand{\newtcbenvironment}[2]{
    \tcolorboxenvironment{#1}{#2, enhanced, breakable, sharp corners,leftrule=2pt, rightrule=0pt, toprule=0pt, bottomrule=0pt}
    \tcolorboxenvironment{#1*}{#2, enhanced, breakable, rounded corners,leftrule=2pt, rightrule=0pt, toprule=0pt, bottomrule=0pt}
}

% 下面就是为前面的各种定理环境加上盒子,参数是盒子的边框颜色 `colframe` 和背景色 `colback`
%
% 具体颜色如下表
%
% |            环境名             |   盒子边框颜色    |    盒子背景色    |
% | :---------------------------: | :---------------: | :--------------: |
% |   `theorem`, `proposition`    |    RoyalPurple    |  RoyalPurple!8   |
% | `corollary`, `lemma`, `claim` |     NavyBlue      |    SkyBlue!8     |
% |         `definition`          |    ForestGreen    |  ForestGreen!5   |
% |           `example`           |     RawSienna     |   RawSienna!5    |
% |           `problem`           | WildStrawberry!30 | WildStrawberry!5 |
%
% 说明:
%
% - 这里采用 `xcolor` 宏包所提供的标准颜色,`xx!n`代表将颜色 `xx` 以 `n%` 比例和白色混合得到的浅颜色。
% - 为了避免颜色过多,对语义类似的环境合并采用相同的盒子颜色。

\newtcbenvironment{theorem}{colframe=RoyalPurple, colback=RoyalPurple!8}
\newtcbenvironment{proposition}{colframe=RoyalPurple, colback=RoyalPurple!8}
\newtcbenvironment{corollary}{colframe=NavyBlue, colback=SkyBlue!8}
\newtcbenvironment{lemma}{colframe=NavyBlue, colback=SkyBlue!8}
\newtcbenvironment{claim}{colframe=NavyBlue, colback=SkyBlue!8}

\newtcbenvironment{definition}{colframe=ForestGreen, colback=ForestGreen!5}
\newtcbenvironment{example}{colframe=RawSienna, colback=RawSienna!5}
\newtcbenvironment{problem}{colframe=WildStrawberry!30, colback=WildStrawberry!5}


\title{Deep Reinforcement Learning for Multiobjective Optimization}
\author{Chen Huaneng}
\date{\today}

\begin{document}

\maketitle

\section{Deep Reinforcement Learning Based Multiobjective Optimization Algorithm (DRL-MOA)}

这个研究基于两个关键背景:

\begin{itemize}
    \item 多目标优化问题(Multiobjective Optimization Problem, MOP)的基本困境:传统方法(如 NSGA-II, MOEA/D)通过迭代更新种群寻找 Pareto 最优解,但面对大规模问题时(如 200 城市的多目标旅行商问题 MOTSP)时,迭代次数多、计算效率低,且问题稍有变化(如城市位置微调)就需要重新计算;
    \item 深度强化学习(Deep Reinforcement Learning, DRL)的优势:DRL 能通过试错学习训练一个“黑箱模型”,训练后只需要一次前向传播就能输出解,无需迭代,且泛化能力强(能够处理未见过的问题实例)。
\end{itemize}

文章\cite{liDeepReinforcementLearning2021}提出的 DRL-MOA 框架,本质是用“分解思想”(来自 MOEA/D\cite{zhangMOEAMultiobjectiveEvolutionary2007})拆解多目标优化问题,用“DRL + 神经网络”(来自 Pointer Network\cite{vinyalsPointerNetworks2015}、Actor-Critic\cite{nazariReinforcementLearningSolving2018,belloNeuralCombinatorialOptimization2017})求解每个子问题,再用“参数迁移”加速训练。

\section{General Framework}

通用框架是 DRL-MOA 的“骨架”,解决了“如何将多目标问题转化为 DRL 可处理的单目标问题”和“如何高效训练多个子问题的模型”这两个核心问题。分为分解策略和邻域参数迁移策略两部分。

\subsection{Decomposition Strategy}

\textit{Decomposition Strategy:} 文章中采用 weighted sum approach \cite{miettinen1999nonlinear} 进行多目标优化问题的分解,也可以采用其他 scalarizing methods,比如 Chebyshev 和 the penalty-based boundary intersection (PBI) method \cite{7572016,7390047}。首先,生成一组均匀分布的权重向量(uniformly spread weight vector)$\lambda^1, \lambda^2, \ldots, \lambda^N$,其中 $N$ 为子问题的数量,比如对于双目标问题( $M = 2$),可以取权重向量为 $(1,0), (0.9,0.1), \ldots, (0,1)$,每个向量表示对不同目标的“重视程度”。对第 $j$ 个权重向量 $\lambda^j = (\lambda^j_1, \lambda^j_2, \ldots, \lambda^j_M)^{\mathrm{T}}$,$M$ 表示目标函数的个数,通过 weighted sum approach,可以将 MOTSP 分解为 $N$ 个单目标优化子问题(scalar optimization subproblems)。第 $j$ 个子问题的目标函数为:

\begin{equation}
    \min g^{ws}(x\mid \lambda^j_i) = \sum_{i=1}^{M} \lambda^j_i f_i(x)\label{eq:obj}
\end{equation}

其中 $f_i(x)$ 是原 MOP 的第 $i$ 个目标函数,$g^{ws}(x\mid \lambda^j_i)$ 是第 $j$ 个子问题的“加权和成本”(单目标)。 

分解后每个子问题的解都是原 MOP 的 Pareto 最优解,这是因为权重向量的不同权衡,使得每个子问题的最优解对应 PF(Pareto Front)上的一个“权衡点”。通过将 MOP 分解成子问题,可以将每个子问题的“加权和成本”作为 DRL 的 “奖励信号(比如奖励 = 负的加权和成本,因为 DRL 通常最大化奖励,而 MOP 需要最小化成本)。这样就通过将 MOP 拆解为多个标量子问题,每个子问题对应一个“权重向量”,求解所有子问题的解就可以组成 PF。

\subsection{Neighborhood-Based Parameter-Transfer Strategy}

\textit{Neighborhood-Based Parameter-Transfer Strategy:}采用邻域参数迁移的策略的核心在于,如果每个子问题都“从头训练”一个神经网络,计算量会非常大($N$ 个子问题需要 $N$ 次独立训练)。但文章根据公式\cref{eq:obj}和Zhang的研究\cite{zhangMOEAMultiobjectiveEvolutionary2007}发现,相邻权重向量对应的子问题,其最优解和最优模型参数非常相似,比如在双目标问题中,权重向量为 $(0.8,0.2)$和$(0.7,0.3)$ 的子问题对于目标的权衡接近,最优路径和模型参数也接近。因此,借鉴 MOEA/D 的“邻域更新”思想\cite{zhangMOEAMultiobjectiveEvolutionary2007},提出了邻域参数迁移策略,即用前一个子问题的最优模型参数,作为当前子问题的初始参数,避免从头训练,减少计算成本。

其具体过程为:假设已经训练好第 $i - 1$ 个子问题的最优模型参数 $[w_{\lambda^{i-1}}^*, b_{\lambda^{i-1}}^*]$($w$ 为权重,$b$ 为偏置),在训练第 $i$ 个子问题时,使用 $[w_{\lambda^{i-1}}^*, b_{\lambda^{i-1}}^*]$ 作为初始参数($[w_{\lambda^i}, b_{\lambda^i}] = [w_{\lambda^{i-1}}^*, b_{\lambda^{i-1}}^*]$)进行训练。然后在此基础上用 Actor-Critic 进行微调,快速收敛到第 $i$ 个子问题的最优参数 $[w_{\lambda^i}^*, b_{\lambda^i}^*]$。重复该过程,直到所有 $N$ 个子问题都训练完毕。

邻域参数迁移策略的优势在于无需为每个子问题初始化随机参数,从而减少了收敛时间,降低了训练复杂度;同时,由于相邻子问题的模型参数平滑过渡,避免 PF 上出现“跳跃”的解,保证了解的一致性和多样性。

\subsection{Pseudo Code of General Framework of DRL-MOA}

DRL-MOA 的通用框架伪代码如\cref{alg:DRL-MOA}所示。每个子问题的训练核心是 Actor-Critic 算法,负责将子问题的“加权和成本”转化为模型的优化信号。训练完成之后,对于新的 MOP 实例,只需要一次前向传播(forward propagation)就能得到对应的 Pareto 最优解,无需重新训练。

\begin{algorithm}[H]
    \KwIn{The model of the subproblem $\mathcal{M} = [\mathbf{w}, \mathbf{b}]$, weight vectors $\lambda^1, \dots, \lambda^N$}
    \KwOut{The optimal model $\mathcal{M}^* = [\mathbf{w}^*, \mathbf{b}^*]$}
    \SetAlgoLined
    \SetNoFillComment
    \vspace{3mm}
    $[\omega_{\lambda^1}, \mathbf{b}_{\lambda^1}] \leftarrow \text{Random\_Initialize}$\;
    \For{$i \leftarrow 1$ \KwTo $N$} {
        \uIf{$i == 1$} {
            $[\omega_{\lambda^1}^*, \mathbf{b}_{\lambda^1}^*] \leftarrow \text{Actor\_Critic}([\omega_{\lambda^1}, \mathbf{b}_{\lambda^1}], g^{\text{ws}}(\lambda^1))$\;
        }
        \Else {
            $[\omega_{\lambda^i}, \mathbf{b}_{\lambda^i}] \leftarrow [\omega_{\lambda^{i-1}}^*, \mathbf{b}_{\lambda^{i-1}}^*]$\;
            $[\omega_{\lambda^i}^*, \mathbf{b}_{\lambda^i}^*] \leftarrow \text{Actor\_Critic}([\omega_{\lambda^i}, \mathbf{b}_{\lambda^i}], g^{\text{ws}}(\lambda^i))$\;
        }
    }
    \Return $[\mathbf{w}^*, \mathbf{b}^*]$\;
    \tcc{Given inputs of the MOP, the PF can be directly calculated by $[\mathbf{w}^*, \mathbf{b}^*]$.}
    \caption{General Framework of DRL-MOA}
    \label{alg:DRL-MOA}
\end{algorithm}

\section{Modeling the Subproblem of MOTSP}

文章通过 MOTSP 实例,展示了如何将具体的 MOP 转化为 DRL-MOA 框架中的子问题,这部分分为 MOTSP 问题的定义、模型结构(经过修改后的 Pointer Network\cite{nazariReinforcementLearningSolving2018})和 训练方法(Actor-Critic\cite{nazariReinforcementLearningSolving2018,belloNeuralCombinatorialOptimization2017})三部分。

文章的实验实例是多目标旅行商问题 MOTSP:The multiobjective traveling salesman problem (MOTSP), where given $n$ cities and $M$ cost functions to travel from city $i$ to $j$, one needs to find a cyclic tour of the $n$ cities, minimizing the $M$ cost functions. 

文章为了将输入 $X$ 映射到输出 $Y$,用概率链式法则(probability chain rule)将排列 $Y$ 的概率分解为条件概率的乘积:

\begin{equation}
    P(Y \mid X) = \prod_{t=1}^{n} P(\rho_{t + 1} \mid \rho_1, \rho_2, \ldots, \rho_{t}, X_t)
\end{equation}

首先,随机选择一个城市作为起点 $\rho_1$,然后在每一步 $t = 1, 2, \ldots$ 过程中,基于当前已选择的城市 $\{\rho_1, \rho_2, \ldots, \rho_t\}$ 在剩余未选择的城市 $X_t$ 中选择下一个城市 $\rho_{t + 1}$,在选择完一个城市后,将其从 $X_t$ 中移除,直到所有城市都被选择完毕。

文章中采用的类似于 Nazari 等人\cite{nazariReinforcementLearningSolving2018} 的 Pointer Network 的基本结构是 Sequence-to-Sequence 模型\cite{sutskeverSequenceSequenceLearning2014}。Sequence-to-Sequence 模型由两层 RNN 组成,分别是 Encoder 和 Decoder。Encoder RNN 负责将输入序列编码为隐藏状态(hidden state),Decoder RNN 则基于隐藏状态生成输出序列。即 Encoder RNN 将输入序列转化成一个上下文向量(code vector),然后 Decoder RNN 基于该上下文向量逐步生成输出序列。

\subsection{Formulation of MOTSP}

One needs to find a tour of $n$ cities, that is, a cyclic permutation $\rho$, to minimize $M$ different cost functions simultaneously. 

\begin{equation}
    \min z_k(\rho) = \sum_{i=1}^{n - 1} c^k_{\rho(i), \rho(i + 1)} + c^k_{\rho(n), \rho(1)}, \quad k = 1, 2, \ldots, M
\end{equation}

where $c^k_{\rho(i),\rho(i + 1)}$ is the $k$-th cost of traveling from city $\rho(i)$ to $\rho(i + 1)$. The cost functions may, for example, correspond to tour length, safety index, or tourist attractiveness in practical applications.

\subsection{Model}

文章采用修改后的 Pointer Network\cite{vinyalsPointerNetworks2015} 作为解决 MOTSP 子问题的神经网络模型。模型可以分为输入和输出结构、编码器(Encoder)、解码器(Decoder)和注意力机制(Attention Mechanism)四部分。

\subsubsection{Input and Output Structure}

模型的输入为 $X = \{s^i, i = 1, 2, \ldots, n\}$,其中 $n$ 为城市的数量。每个 $s^i$ 由一个元组(tuple)$\{s^i = (s^i_1, s^i_2, \ldots, s^i_M)\}$表示,$s^i_j$为城市 $i$ 在第 $j$ 个目标下用于计算成本的特征。比如 $s^i_1 = (x_i, y_i)$ 表示城市 $i$ 的二维坐标(用于计算距离),$s^i_2$ 可以表示城市 $i$ 的安全指数等。模型的输出为一个城市的排列 $Y = \{\rho_1, \rho_2, \ldots, \rho_n\}$。

\subsubsection{Encoder}

\textit{Encoder:} Encoder 的作用是将输入的城市特征 $X$ 编码为一个高维向量,方便 Decoder 使用。由于城市的位置是无序的,在输入结构中城市之间的顺序是没有任何意义的,因此文章中没有采用 RNN 作为 Encoder(比如 long short-term memory (LSTM),因为 RNN 会引入不必要的“顺序偏见”),而且 1-D 卷积层参数是共享的,计算更快,且对城市数量的泛化性更强,因此使用了 a simple embedding layer (the 1-D convolution layer) 来将输入 $X$ 编码到一个高维向量(high-dimensional vector)空间中\cite{nazariReinforcementLearningSolving2018}(文章中的超参数 $d_h = 128$)。

\subsubsection{Decoder}

\textit{Decoder:} Decoder 的作用是按顺序生成城市排列 $Y$(或者说一个城市的选择序列索引)。由于在 Decoder 中需要总结之前选择的城市信息 $\rho_1, \ldots, \rho_t$,然后进行下一步的城市选择 $\rho_{t + 1}$,因此 Decoder 中需要使用 RNN,文章中使用 the gated recurrent unit (GRU)\cite{choLearningPhraseRepresentations2014},这种 RNN 结构比 LSTM 的参数更少但是性能相似(原始的 Pointer Network 使用的是 LSTM\cite{nazariReinforcementLearningSolving2018})。文章中的 RNN 并不是直接用于生成下一个城市的选择,而是用 RNN Decoder 在解码到 $t$ 时的隐藏状态(hidden state)$d_t$ 来存储之前选择的城市信息 $\rho_1, \ldots, \rho_t$。通过结合 $d_t$ 和 Encoder 的输出 $e_1,\ldots, e_n$ 用注意力机制(Attention Mechanism)来计算下一个城市选择的条件概率 $P(\rho_{t + 1} \mid \rho_1, \ldots, \rho_t, X_t)$。

\subsubsection{Attention Mechanism}

\textit{Attention Mechanism:} 直观上来看,注意力机制计算每个输入 $e_i$ 对下一次解码步骤 $t$ 的相关性(relevance),最相关的输入 $e_i$ 会被赋予更高的权重(the most relevant one is given more attention)并且更有可能被选择为下一个访问的城市。计算的公式如公式\cref{eq:attention}所示。

\begin{equation}
    u_j^t = v^\mathrm{T} \tanh(W_1 e_j + W_2 d_t), \quad j \in (1, 2, \ldots, n)\label{eq:attention}
\end{equation}

其中 $W_1, W_2$ 和 $v$ 是需要学习的参数,$e_j$ 是 Encoder 的输出,$d_t$ 是 Decoder 在时间步 $t$ 的隐藏状态。通过对每个未选择的城市 $j$ 用 GRU 的隐藏状态 $d_t$ 和 Encoder 的输出 $e_j$ 计算相关性得分 $u_j^t$,$u_j^t$越大,说明城市 $j$ 越适合作为下一步选择的城市。然后通过 softmax 函数将相关性得分转化为概率分布,即将 $u_1^t, u_2^t, \ldots, u_n^t$ 归一化(normalize)为概率。

\begin{equation}
    P(\rho_{t + 1} \mid \rho_1, \ldots, \rho_t, X_t) = \text{softmax}(u^t)\label{eq:softmax}
\end{equation}

在选择下一个城市 $\rho_{t + 1}$ 时,文章采用贪心策略(greedy decoder)选择概率最大的城市作为下一个访问的城市。在训练过程中,文章采用采样策略(sampling),从概率分布中采样下一个城市 $\rho_{t + 1}$,以增加探索性。

\subsection{Training Method}

文章中采用 Actor-Critic 算法训练子问题的模型\cite{nazariReinforcementLearningSolving2018, belloNeuralCombinatorialOptimization2017}。Actor-Critic 算法的伪代码如\cref{alg:actor-critic}所示。

\begin{algorithm}[H]
    \KwIn{$\theta, \phi \leftarrow$ Initialized parameters given in \cref{alg:DRL-MOA}}
    \KwOut{The optimal parameters $\theta, \phi$}
    \SetAlgoLined
    \SetNoFillComment
    \vspace{3mm}
    \For{iteration $\leftarrow 1, 2, \dots$} {
        generate $T$ problem instances from $\{\Phi_{\mathcal{M}_1}, \dots, \Phi_{\mathcal{M}_M}\}$ for the MOTSP\;
        \For{$k \leftarrow 1, \dots, T$} {
            $t \leftarrow 0$\;
            \While{not terminated} {
                select the next city $\rho_{t+1}^k$ according to $P(\rho_{t+1}^k | \rho_1^k, \dots, \rho_t^k, X_t^k)$\;
                Update $X_t^k$ to $X_{t+1}^k$ by leaving out the visited cities\;
            }
            compute the reward $R^k$\;
        }
        $d\theta \leftarrow \frac{1}{N} \sum_{k=1}^N \left( R^k - V(X_0^k; \phi) \right) \nabla_\theta \log P(Y^k | X_0^k)$\;
        $d\phi \leftarrow \frac{1}{N} \sum_{k=1}^N \nabla_\phi \left( R^k - V(X_0^k; \phi) \right)^2$\;
        $\theta \leftarrow \theta + \eta d\theta$\;
        $\phi \leftarrow \phi + \eta d\phi$\;
    }
    \caption{Actor--Critic Training Algorithm}
    \label{alg:actor-critic}
\end{algorithm}

Actor-Critic 算法分为两个主要部分:Actor 网络和 Critic 网络。Actor 网络就是前面介绍的修改后的 Pointer Network,负责给出选择下一个城市的概率分布 $P(\rho_{t + 1} \mid \rho_1, \ldots, \rho_t, X_t)$,参数记为 $\theta$。Critic 网络用于评估当前问题实例的价值(evaluates the expected reward given a specific problem state),参数记为 $\phi$,结构和 Encoder 一致(1-D 卷积层),输入是问题实例,输出是一个标量(价值估计)。

训练是一个无监督的过程(unsupervised),目标是最大化期望奖励,即最小化加权和成本。训练的核心是策略梯度下降,首先通过从分布 $\{\Phi_{\mathcal{M}_1}, \ldots, \Phi_{\mathcal{M}_M}\}$ 中采样 $T$ 个 MOTSP 实例,其中,$\mathcal{M}$ 表示城市的不同输入特征(比如城市坐标、安全指数等)。然后对于每个实例,通过当前参数为 $\theta$ 的 Actor 网络生成一个城市排列 $\rho^k$,计算对应的奖励 $R^k$。然后用策略梯度(policy gradient)更新 Actor 网络的参数 $\theta$\cite{kondaActorCriticAlgorithms1999}。其中,$V(X_0^n; \phi)$ 是 Critic 网络对问题实例 $n$ 的价值估计(reward approximation)。然后通过最小化“真实奖励”和 Critic 估计价值之间的平方差来更新 Critic 网络的参数 $\phi$。重复该过程,直到参数收敛。




\bibliography{references}

\end{document}
